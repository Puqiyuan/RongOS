\documentclass{wx672article} % $HOME/texmf/tex/latex/wx672article.cls

% wx672{common,fonts,bib} loaded in wx672a.sty
\usepackage{wx672cjk}

\title{作品简介}
\author{蒲启元 \\
\emph{pqy7172@gmail.com}}

\begin{document}

\maketitle{}
操作系统管理着计算机的硬件和软件资源,它是向上层应用软件提供服务(接口)的核心系统软件,这
些服务包括进程管理,内存管理,文件系统,网络通信,安全机制等。操作系统的设计与实现则是软件
工业的基础。为此,在国务院提出的《中国制造2025》中专门强调了操作系统的开发。但长期以来,操
作系统核心开发技术都掌握在外国人手中,技术受制,对于我们的软件工业来说很不利。本项目从零开
始设计开发一个简单的操作系统,所谓从零开始是指它直接面对硬件,不依赖任何三方库或者建立在任
何已有的内核基础上,其中包括boot loader,中断,内存管理,图形接口,多任务等功能模块,以及
能运行在这个系统之上的几个小应用程序。尽管这个系统很简单,但它是自主开发操作系统的一次尝试。

本系统的实现过程涵盖了主要的操作系统中的概念,比如内存管理,进程管理,多任务等,本系统采用
的工程实现都能找到对应的操作系统理论。实现了对于鼠标移动,键盘输入的支持。本系统是基于32位
的。保护操作系统也是一个十分重要的问题,保护的方面包括应用程序不能破坏操作系统本身,也包括
应用程序之间不能互相干扰,尽管这些问题比较复杂,但是本系统也支持操作系统的保护。本系统可以
读取FAT表。

在设计上按照操作系统开发理论,本系统具有较为清晰的层次与模块。一个操作系统应该提供一些API
以方便应用程序员开发他们的程序,本系统也提供了几个简单的API,有几个应用是建立在这些API的基
础之上,这证明了本操作系统可以良好的运行应用,只要这些应用遵守本操作系统的API。

开发一个操作系统不是一件容易的事,理论与实践上都得有充分准备。操作系统中许多模块都是相互关
联的,这些关系还相当复杂。随这个作品一起的还有大量的图解,这有助于理解操作系统中的抽象概念。

项目地址:https://github.com/Puqiyuan/RongOS。


\end{document}

%%% Local Variables:
%%% mode: latex
%%% TeX-master: t
%%% End:
