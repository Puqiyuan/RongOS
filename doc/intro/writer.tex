\documentclass{wx672article} % $HOME/texmf/tex/latex/wx672article.cls

% wx672{common,fonts,bib} loaded in wx672a.sty
\usepackage{wx672cjk}

\title{作者简介}
\author{蒲启元 \\
\emph{pqy7172@gmail.com}}

\begin{document}

\maketitle{}

蒲启元,男,22岁。西南林业大学计算机科学与技术专业学生。热爱操作系统开发。校级、省级三好学
生,优秀毕业生,优秀毕业论文(设计),泰国交换生。Github主页:https://github.com/Puqiyuan。

具有较强的自学能力,大学期间培养了使用Google快速解决问题的能力,学习新东西较快,由于使用
Google英文,英文阅读沟通能力较强(CET6,考研英语一70分,满分100)。

具有三年Debian Linux使用配置经验,熟悉Linux环境,有一定的Bash Shell编程经验。善于使用
Makefile,git,vim,emacs。编程,逻辑能力较优。耐心毅力都不错,大学期间每周坚持十公里⻓跑。
解决技术问题很多时候更靠耐心和毅力。热爱学习,大学期间多次成绩排名前列,主要的专业课程比如
操作系统,数据结构,组成原理,网络等都是90分以上。

具有良好的编码技能,比如清华清橙在线判题系统的解题记录:
https://github.com/Puqiyuan/Tsinsen\_ACM。这个项目是独立自主
完成清橙编程题目。目前所完成的题目都是满分通过,其中某些题目的通过率较低,最低只有16\%。所
有题解代码最大的特色就是我只依赖了C标准库中少量的几个基本库函数比如printf,malloc等。当然
实际生产产品时不提倡这样做,但在学习阶段我的观点还是较多的自己写代码,以训练自己的编码技巧。
其它类似此的题解练习还有:https://github.com/Puqiyuan/URI\_ACM。



\end{document}

%%% Local Variables:
%%% mode: latex
%%% TeX-master: t
%%% End:
