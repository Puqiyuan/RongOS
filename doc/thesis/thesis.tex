\documentclass{swfcthesisp}
\usepackage{xcolor}
\usepackage{hyperref}
\usepackage[utf8]{inputenc}
\usepackage{listings}
\usepackage{verbatim}
\addbibresource{thesis.bib}    % 参照教程自己去写一个.bib文件

\begin{document}
\cite{os_wiki}
\cite{china_2025}
\cite{30_os}

\Title{RongOS --- 一个简单操作系统的实现}
\Author{蒲启元}
\Advisor{王晓林}
\AdvisorTitle{讲师}
\AdvisorInfo{王晓林,男,49岁,硕士,讲师,毕业于英国格林尼治大学,分布式系统专业,现任西
  南林业大学计信学院教师,执教Linux、操作系统、网络技术等方面的课程,有丰富的Linux教学和系
  统管理经验。}
\Month{六}
\Year{二〇一八}
\Univ{西南林业大学}
\School{计算机与信息科学学院}      %院系名称
\Subject{计算机科学与技术专业}    %专业名称(比如 电子信息工程专业)
\Docname{本科毕业(设计)论文}    %本科?研究生?
\Abstract{操作系统管理着计算机的硬件和软件资源,它是向上层应用软件提供服务(接口)的核心系
  统软件,这些服务包括进程管理,内存管理,文件系统,网络通信,安全机制等。操作系统的设计与
  实现则是软件工业的基础。为此,在国务院提出的《中国制造2025》中专门强调了操作系统的
  开发。但长期以来,操作系统核心开发技术都掌握在外国人手中,技术受制,对于我们的软件工业来
  说很不利。本文从零开始设计开发一个简单的操作系统,包括boot loader,中断,内存管理,图
  形接口,多任务,以及在这个系统上的几个小应用等。尽管这个系统很简单,但它为自主开发操作系
  统做了一个小小的尝试。}
\Keywords{操作系统,进程,内存,中断,boot loader}
\Acknowledgments{首先我想感谢我的老师,王晓林。大学期间,他给了我很多指导,包括专业方面和
  上大学的意义等。很多时候,他对学生的要求看起来都是不近情理的,但正是通过这个“痛苦”的过程,
  我锻炼了坚强的意志,和战胜困难的信心。谢谢你,王老师。我最想感谢的是我的女友,她容忍我在
  完成这个设计时的很多个夜晚不陪她,给我支持,鼓励我,不抱怨。所以我愿意把这个简单操作系统
  命名为RongOS, 蓉便是她名字的最后一个字。谢谢你,我最亲爱的。}
\enTitle{RongOS --- A simple OS implementation}
\enAuthor{Qiyuan Pu}
\enUniv{Southwest Forestry University}
\enSchool{School of Computer and Information Science} %英文院系名称
\enAbstract{Operating system manages the sources of hardware, it lies in the
  core of the system software and provides services(interfaces) to upper applications. These
  service including process management, memory management, file system, network
  communication, security mechanism etc. Operating system development is
  the foundation and core of software industry. Therefore, <<Made in China 2025>>
  emphasize the development of operating system that put forward by The State Council of China. For
a long time, however, the OS kernel development technology grasped in the hand of foreigner,
it's bad for our software industry cause of limited technology. So this article will
design and develop a simple operating system, including boot loader, interrupt, memory
management, graphic interface, multitasking, and some little applications based on this
system. In spite of the simplicity of this system, it's a small trying for autonomous
development operating system.}
\enKeywords{operating system, boot loader, process, interrupt, memory management}

%%% 下面六行不要动!
\makepreliminarypages% 封面
\frontmatter          
\tableofcontents     % 目录
\listoffigures       % 插图目录
\listoftables        % 表格目录
\mainmatter
\chapter{Introduction}

\section{Background}
\label{sec:background}


\section{Preliminary Works}

\subsection{Development Environment}
\label{sec:devel-envir}

Operating System: Debian 4.11.0-1-amd64 \\
\hspace*{0.8cm}Debug System: QEMU emulator version 2.8.1(Debian 1:2.8+dfsg-7)\\
\hspace*{0.8cm}Emacs version: GNU Emacs 25.2.2

\subsection{Tools}
\label{sec:tools}

Some tools used to develop RongOS, see
tools.\footnote{\url{https://github.com/Puqiyuan/RongOS/tree/master/Tools}}.

\subsection{Install}
\label{sec:install}

Debian System: there is a small
tutorial.\footnote{\url{http://cs2.swfc.edu.cn/~wx672/lecture_notes/linux/install.html}}\\
\hspace*{0.8cm}QEMU, for my x86\_64 architecture: 
\begin{lstlisting}[language=bash]
     $ sudo apt-get install qemu-system-x86_64
\end{lstlisting}

Note that the tools is exe formate, so on Debian system, you need to install wine:
\begin{lstlisting}
     $ sudo apt-get update
     $ sudo apt-get install wine
\end{lstlisting}

Maybe you also need to add i386 architecture cause of AMD64 on your machine to use \hspace*{0.8cm}these tools:
\begin{lstlisting}
     $ sudo dpkg --add-architecture i386
     $ sudo apt-get update
\end{lstlisting}


\begin{comment}
\subsection{表格示例}

下面是一个表格的例子:

\begin{table}[!ht]
  \centering
  \begin{tabular}{|r|c|l|}
    \hline
    Hello&world&Hello, world!\\
    \hline
    Hello&world&Hello, world!\\
    \hline
  \end{tabular}
  \caption{表格示例}
  \label{tab:hello}
\end{table}
\end{comment}


\chapter{Design}

\section{Top Level Design}
\label{sec:top-level-design}

\section{Detailed Design}
\label{sec:detailed-design}

\subsection{Boot Loader}
\label{sec:boot-loader}
This is working flow of boot loader:
\begin{figure}[!ht]
  \centering
  \includegraphics[width=.5\textwidth]{design.pdf}
  \caption{Working Flow of Boot Loader}
  \label{fig:working-flow-boot-loader}
\end{figure}

The instructions of boot loader saved in C0-H0-S1 of floppy, the first cylinder, head 0,
the first sector, total 512 byte. These instructions end with 0x55 0xaa, so BIOS will load
C0-H0-S1 to memory, then the instructions in C0-H0-S1 will load C0-H0-S2 --- C9-H1-S18,
total $10*2*18*512=184320byte=180KB$(including boot sector, C0-H0-S1) to main memory.

\subsection{32-bit Mode and Import C Codes}
\label{sec:32-bit-mode}




\chapter{Implementation}

\section{Boot Loader}

\subsection{Chose Disk}
\label{sec:chose-disk}

There are many ways to boot a operating system, from hard disk, USB, floppy disk etc. I
chose floppy disk, although it is out of date. For my purpose is that develop a simple
operating system, pay my attention on how to development. The structure of floppy disk is
simple and for my simple operating system it's enough.

\subsection{The Structure of Floppy Disk}
\label{sec:struct-floppy-disk}

This picture show the inside of floppy disk:
\begin{figure}[!ht]
  \centering
  \includegraphics[width=.5\textwidth]{../figs/bootLoader/flpy1.png}
  \caption{Floppy Disk Structure}
  \label{fig:flpy1.png}
\end{figure}

The floppy store information in two sides. There are $80$ cylinders from the outermost to
the core in each side, numbering $0$, $1$, ..., $79$. The head can assign be $0$ or $1$,
representing two sides of floppy. When specify head number and cylinder number, forming a
ring, named track in jargon. The track is large so we divide it to 18 small parts, named
sector. A sector can store $512$ byte. So the capacity of a floppy is:

$$18 * 80 * 2 * 512 = 1474560 Byte = 1440 KB.$$

The IPL(Initial Program Loader) in C0-H0-S1(cylinder 0, head 0, sector 2), and the next
sector is C0-H0-S2.


\subsection{Flowchart of Boot Loader}
The following is the flowchart of boot loader:
\label{sec:flowch-boot-load}
\begin{figure}[!ht]
  \centering
  \includegraphics[width=1\textwidth]{../figs/FlowchartTex/1/flowchartp.pdf}
  \caption{Flowchart of Boot Loader}
  \label{fig:flowchart-of-boot-loader}
\end{figure}

Firstly, the boot sector display some boot information, when $al=0$, the null
character of boot information hit. Interrupt $0x10$ is used for show a character.
\par
Then jump to load C0-H0-S2, $ax$ register saved the
address where beginning puts the sectors from floppy. And preparing parameters for
interrupt $0x13$ in registers. The $0x13$ interrupt used for read sector from floppy to
memory.
\par If there is a carry, representing some thing wrong when read floppy, so reset the
registers and try again read floppy, until five times trying. Register $si$ is a counter. If
no carry, jump to next segmentation, as one sector read to memory already, the address
space should increase 512 byte. Then sector number($cl$ register) added 1 and compare it to 18, if it's
smaller than 18, jump to $readloop$, read the next sector. If the value of $cl$ register
bigger or equal to than 18, meaning that one track 18 sector in this side of floppy read already, then
reversed the head, add 1 to $dh$ register. If the value of $dh$ register after adding
larger than or equal to 2, it's saying the original head is 1, one track of two sides read
already. Otherwise the value of $dh$ register smaller than 2, read this side indicating by
$dh$ register, jump to $readloop$ segmentation.

\par So the next step is moving a cylinder, add 1 to register $ch$. Otherwise the
value of $dh$ register smaller than 2, read this side indicating by $dh$ register, jump to
$readloop$ segmentation. After $ch$ register add 1, if it's smaller than 10, jump to
$readloop$, otherwise end loading floppy to memory process, for we only load ten cylinders
of floppy.



\subsection{Codes and Comments of Boot Loader}
\label{sec:codes-comments}


\inputminted[linenos=true]{nasm}{../../src/06day/RongC/ipl10.nas}

\subsection{Running Result}
\label{sec:running-result}
\begin{figure}[!ht]
  \centering
  \includegraphics[width=1.1\textwidth]{iplRes.png}
  \caption{Running Result of Boot Loader}
  \label{fig:iplRes}
\end{figure}



\section{32-bit Mode and Import C Codes}


\section{Screen Display and Text}

\section{Control Mouse}


\section{Memory Management}

\section{Making Window }

\section{Timer}

\section{Multitasking}

\section{Command Line Window}

\section{API}

\section{OS Protection}

\section{Graphics Processing}

\section{Window Operation}

\section{Application Protection}

\section{File Operation}

\section{Some Applications}

\section{Prospects and Shortages}

%%% 正文部分到此结束。下面是『参考文献』、『指导教师简介』、『鸣谢』、『附录』

%% 不要动下面四行!
\Appendix{}
\printbibliography[heading={bibintoc},title={参考文献}] % 输出参考文献
\advisorinfopage{}                 % 输出指导教师简介
\acknowledgmentspage{}             % 输出鸣谢

%%% 下面是附录部分,可以没有。

%\chapter{我也不知道为什么要写附录} %附录一

%可以参考模版目录中的 appendix.tex 文件来写。

%\chapter{主要程序代码} %附录二

% 插入程序代码

% 也可以这样
\end{document} % 结束。不要动下面几行!

%%% Local Variables:
%%% mode: latex
%%% TeX-master: t
%%% End:
